\documentclass[dvipdfmx]{jlreq}

	\usepackage{enumerate}
	\usepackage{amsthm}
	\usepackage{amsmath}
	\usepackage{amssymb}
	\usepackage{amsfonts}

	\title{情報科学応用 第4回課題}
	\author{細川 夏風}
	\date{\today}

\begin{document}

	\maketitle

	\section{問題A}
		$m, n, k \in \mathbb{Z}$	
		\begin{enumerate}[(1)]
			\item $2 > 0 \land \exists k(-2020 = 2k)$
			\item $\forall m \exists k(0 = mk)$
			\item $\forall m (m > 0 \land \exists k(0 = mk))$
			\item $\exists n ((0 > 0) \land \exists k(n = 0k))$
			\item $\forall m \exists k(-1 = mk)$
		\end{enumerate}

	\section{問題B}
		\begin{enumerate}[(1)]
			\item この命題に対して、論理式$2 > 0 \land \exists k(-2020 = 2k)$を$2 > 0 \land$の部分と$\exists k(-2020 = 2k)$に分けて考える.$2 > 0$に付いて2は0より大きいので真.$\exists k(-2020 = 2k)$について、$k = -1010$のとき成り立つため真である.よってこの命題は真である.
			\item この命題の論理式$\forall m \exists k(0 = mk)$から考える.任意の整数となるような$m'$をとったとき、$k = 0$をとると$m' \times 0$は$0$になるためこの命題は真である.
			\item この命題の論理式$\exists m(m \leq 0 \lor \forall k(0 \neq mk))$から考える.このとき、$m = 0$であればこの命題は成り立つ.よって否定命題が真であるため順命題は偽である.
			\item この命題の論理式$\exists n ((0 > 0) \land \exists k(n = 0k))$から考える.この命題は論理式の前部分の$(0 > 0)$の部分で成り立っていないため、偽である.
			\item この命題の論理式$\forall m \exists k(-1 = mk)$の否定命題$\exists m \forall k (-1 \neq mk)$について考える.任意の整数$k'$をとる.このとき、$m = 0$とすると、$mk' \neq 0$となるため否定命題が成り立つため、順命題は偽である.
		\end{enumerate}

	\section{問題C}
		\begin{proof}
			$P$を数列$a_n$の述語としたとき、$\forall n P(n)$について帰納法を用いて成り立つことを証明する.

			$n = 1$のとき、$10 \times 2 ^ {1 - 1} -3$の解は$7$となるため成り立つ.

			$a_n$のとき、
			\begin{equation}
			\begin{split}
				2(10 \times 2 ^ {n - 1} -3) + 3 &= 10 \times 2 ^ n - 6 + 3 \\
				&= 10 \times 2 ^ n - 3 \\
				&= a_{n+1}
			\end{split}
			\end{equation}
			よって数列$a_n$はすべての自然数nについて成り立つ.
		\end{proof}

		\section*{問題 D の証明}

			\begin{proof}
				
				基底の場合 ($n = 0$)

				左辺と右辺をそれぞれ計算する.

				\[
				\sum_{i=0}^0 i^3 = 0^3 = 0
				\]
				\[
				\left( \sum_{i=0}^0 i \right)^2 = (0)^2 = 0
				\]

				左辺と右辺が一致するため、$P(0)$ は成立する.				

				$n = k$ のとき、次が成り立つと仮定する.
				\[
				\sum_{i=0}^k i^3 = \left( \sum_{i=0}^k i \right)^2
				\]
				つまり、
				\[
				\sum_{i=0}^k i^3 = \left( \frac{k(k+1)}{2} \right)^2
				\]
				が成立すると仮定する.

				$n = k+1$ の場合

				左辺を計算する.
				\[
				\sum_{i=0}^{k+1} i^3 = \sum_{i=0}^k i^3 + (k+1)^3
				\]

				帰納法の仮定を利用すると
				\[
				\sum_{i=0}^{k+1} i^3 = \left( \sum_{i=0}^k i \right)^2 + (k+1)^3
				\]

				\(\sum_{i=0}^k i = \frac{k(k+1)}{2}\) より:
				\[
				\sum_{i=0}^{k+1} i^3 = \left( \frac{k(k+1)}{2} \right)^2 + (k+1)^3
				\]

				次に、右辺を計算する.
				\[
				\left( \sum_{i=0}^{k+1} i \right)^2 = \left( \sum_{i=0}^k i + (k+1) \right)^2
				\]
				\[
				\left( \sum_{i=0}^{k+1} i \right)^2 = \left( \frac{k(k+1)}{2} + (k+1) \right)^2
				\]

				\(\frac{k(k+1)}{2} + (k+1)\) を計算すると
				\[
				\frac{k(k+1)}{2} + (k+1) = \frac{k(k+1) + 2(k+1)}{2} = \frac{(k+1)(k+2)}{2}
				\]

				これを二乗すると
				\[
				\left( \frac{(k+1)(k+2)}{2} \right)^2 = \frac{(k+1)^2(k+2)^2}{4}
				\]

				したがって、右辺は
				\[
				\left( \sum_{i=0}^{k+1} i \right)^2 = \frac{(k+1)^2(k+2)^2}{4}
				\]

				左辺を同様に計算する.
				\[
				\sum_{i=0}^{k+1} i^3 = \left( \frac{k(k+1)}{2} \right)^2 + (k+1)^3
				\]

				\(\left( \frac{k(k+1)}{2} \right)^2 = \frac{k^2(k+1)^2}{4}\) と \((k+1)^3 = \frac{4(k+1)^3}{4}\) をまとめて:
				\[
				\sum_{i=0}^{k+1} i^3 = \frac{k^2(k+1)^2}{4} + \frac{4(k+1)^3}{4}
				\]
				\[
				= \frac{(k+1)^2(k^2 + 4(k+1))}{4}
				\]

				ここで \(k^2 + 4k + 4 = (k+2)^2\) を利用すると
				\[
				\sum_{i=0}^{k+1} i^3 = \frac{(k+1)^2(k+2)^2}{4}
				\]

				右辺と一致した.

				基底の場合 $n=0$ が成立し、帰納法の仮定を用いて $n=k+1$ の場合も成立することを示した.したがって、数学的帰納法により、すべての自然数 $n$ に対して
				\[
				\sum_{i=0}^n i^3 = \left( \sum_{i=0}^n i \right)^2
				\]
				が成立する.

			\end{proof}

\end{document}

